\documentclass[tikz]{standalone}%standalone
%\documentclass[11pt]{ctexart}
%\documentclass〔12pt,dvipdfm]{article}


%\documentclass[a4paper,10pt]{article}

%\documentclass[a4paper,10pt]{ctexart} % instead of `article'

\usepackage{tikz}
\usetikzlibrary{arrows,chains,matrix,positioning,scopes,graphs}

\usepackage{smartdiagram}
\usepackage{forest}
\usepackage{tikz-qtree}

\usepackage{xeCJK}%根据自己的需要加载宏包


%\setmainfont{Caladea}

\makeatletter

% \tikzset{join/.code=\tikzset{after node path={%
% \ifx\tikzchainprevious\pgfutil@empty\else(\tikzchainprevious)%
% edge[every join]#1(\tikzchaincurrent)\fi}}}

% \tikzset{every tree node/.style={align=center,anchor=north}}


\makeatother
%
% \tikzset{>=stealth',every on chain/.append style={join},
%          every join/.style={->}}
% \tikzstyle{labeled}=[execute at begin node=$\scriptstyle,
%    execute at end node=$]
%
\begin{document}

\begin{tikzpicture}
\tikzset{every tree node/.style={align=center,anchor=north}}
\tikzset{edge from parent/.style=
{draw,
edge from parent path={(\tikzparentnode.south)
-- +(0,-8pt)
-| (\tikzchildnode)}}}
\Tree [.指控研究
[.指控原理
    [ 敏\\捷\\指\\控  多\\域\\指\\控  C2\\组\\织\\设\\计  .\\.\\. ] ]
[.指控技术
    [ .态势认知技术
        [ 结\\构\\还\\原 目\\标\\识\\别 意\\图\\理\\解 威\\胁\\分\\析 目\\标\\选\\择 .\\.\\. ] ]
    [ .任务规划技术
        [ 兵\\力\\匹\\配 计\\划\\生\\成 冲\\突\\消\\解 方\\案\\评\\估 临\\机\\规\\划 .\\.\\.  ] ] ]
]
\end{tikzpicture}





% \begin{forest}
% for tree={%
%     edge path={\noexpand\path[\forestoption{edge}] (\forestOve{\forestove{@parent}}{name}.parent anchor) -- +(0,-12pt)-| (\forestove{name}.child anchor)\forestoption{edge label};}
%     }
% [
% AB
% [CD [组织设计] [体系优化[意图识别][态势感知][威胁分析]]]
% [指控技术
% [态势技术[意图识别][态势感知][威\\胁\\分\\析]]
% [规划技术
% [应急规划]
% [XX规划]
% [P]
% [T2]
% [T2*]
% ]
% ]
% ]
% \end{forest}


%     \begin{tikzpicture}[domain=0:5]
%   \draw[very thin,color=gray] (-0.1,-1.1) grid (8.2,8.2);
%   \draw[->] (-0.2,0) -- (8.2,0) node[right] {$x$};
%   \draw[->] (0,-1.2) -- (0,8.2) node[above] {$f(x)$};
%   \draw[color=red]    plot (\x,\x)             node[right] {$f(x) =x$};
%   % \x r 表示弧度
%   \draw[color=blue]   plot (\x,{sin(\x r)})    node[right] {$f(x) = \sin x$};
%   \draw[color=orange] plot (\x,{0.05*exp(\x)}) node[right] {$f(x) = \frac{1}{20} \mathrm e^x$};
% \end{tikzpicture}

% \smartdiagram[circular diagram:clockwise]{Edit,我们,
%   pdf\LaTeX, Bib\TeX/ biber, make\-index, pdf\LaTeX}

% \begin{tikzpicture}
%   \matrix (m) [matrix of math nodes, row sep=3em, column sep=3em]
%     { 0 & A & B  & C  & 0 \\
%       0 & A' & B' & C' & 0 \\ };
%   { [start chain] \chainin (m-1-1);
%     \chainin (m-1-2);
%     { [start branch=A] \chainin (m-2-2)
%         [join={node[right,labeled] {\eta_1}}];}
%     \chainin (m-1-3) [join={node[above,labeled] {我}}];
%     { [start branch=B] \chainin (m-2-3)
%         [join={node[right,labeled] {\eta_2}}];}
%     \chainin (m-1-4) [join={node[above,labeled] {\psi}}];
%     { [start branch=C] \chainin (m-2-4)
%         [join={node[right,labeled] {\eta_3}}];}
%     \chainin (m-1-5); }
%   { [start chain] \chainin (m-2-1);
%     \chainin (m-2-2);
%     \chainin (m-2-3) [join={node[above,labeled] {\varphi'}}];
%     \chainin (m-2-4) [join={node[above,labeled] {\psi'}}];
%     \chainin (m-2-5); }
% \end{tikzpicture}


\end{document}
